Reinforcement learning is a powerful tool that can be used to solve many problems.
These problems can range from extremely simple to extremely complex, and the agents
developed to solve these problems can be just as complex.
There is a need to develop agents that are complex enough to solve a given problem,
but simple enough so they do not require a large amount of resources.
This work makes the use of reinforcement learning agents in a well known
environment, known as Waterworld, and explores the effects of varying the complexity
of the agents.
Multiple algorithms are used to develop these agents, including Q-learning and
Actor-Critic models.
Similarly, the input received by the agents is changed in an attempt to build agents
that still work well, but require less resources.
% TODO: Fill in what is discovered
We hope to discover how to properly design agents in a cost-effective manner so the
agents can solve the problems for which they are designed without costing more
resources than required.

Multicellular organisms depend on vascular systems for nutrient delivery and waste removal\cite{delindavis:bloodevolution}. These vascular networks are formed either through vasculogenesis, a biological process in which scattered vessel precursor cells self-organize to create new networks or through angiogenesis, in which new vessels sprout from the existing vessels.

Both vasculogenesis and angiogenesis are driven primarily by chemotaxis, a mechanism in which cells move in response to a chemical gradient, along with cell-cell adhesion \cite{delindavis:Merks2008ContactInhibited}. While many questions remain, progress in understanding and exploiting both vasculogenesis and angiogenesis is being made from a bioengineering perspective  \cite{Kaully2009VascularizationThe} \cite{Lovett2009Vascularization}. Dahl et al. \cite{delindavis:vascularnetworksinorgandevelopment} successfully implanted tissue-engineered vascular grafts in baboons and dogs. Melero-Martin et al. \cite{delindavis:bioengineeredvascularnetworks} showed that robust development of functional vascular networks is possible \textit{in vivo}.

With additional research in this area, bioengineered cells could be used to form functional vascular networks to create a useful delivery mechanism in synthetic tissues. However, to apply bioengineering approaches to vascular cells, genetic targets need to be identified that when modified, improve the effectiveness of the vascular systems that emerge. Given the vast number of possible targets, a method is required that could assist engineers in identifying those genetic targets.

This paper presents a proof-of-concept genetic algorithm approach to solving this problem. The genetic algorithm acts to modify a computational model of vascular network development embedded within a cell cultivation environment. The search space of the optimization are the values of model parameters which control the mechanisms of the vascular cells such as chemotaxis. Changing these parameters modifies the behavior of individual cells which ultimately influence the spatial organization of the emergent vascular network. To evaluate the quality of this network, the model simulates its operation by determining fluid flow through each vessel, and subsequent nutrient delivery to the cultured cells. The returned fitness value quantifies the total metabolic activity of culture by using bioengineered microbial cells and measuring the total product produced.

The paper is organized as follows. First, the computational model of de novo vascularization is described followed by a description of the genetic algorithm. Next, we present the results which show that genetic algorithms can identify parameter values that improve network quality. There follow details of how the network operation is simulated and the fitness function values calculated. The paper concludes with a discussion of the potential impact of this work and the remaining challenges which need to be solved to operationalize the technique.




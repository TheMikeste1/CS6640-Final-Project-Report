\section{Introduction}\label{sec:introduction}
% What and why
% Problem statement
% Given: find: such that: (solving a problem)
% Question if a methodological study
Reinforcement learning is an extremely powerful tool that is able to solve a multitude
of problems.
One such toy problem is known as Waterworld~\cite{Karpathy2015, Ho2016}, which consists
of an agent chasing food and avoiding poison.
This problem has been expanded by~\cite{Gupta2017} to include multiple agents,
turning it into a multi-agent reinforcement learning problem with both cooperative
and competitive elements.

Waterworld is a fascinating problem with many ways to approach it.
In the modern version provided by the Farama Foundation~\cite{WaterworldDocumentation},
agents receive $8 * \text{number of sensors} + 2$ or
$5 * \text{number of sensors} + 2$ (depending on settings) total observations.
With the default number of sensors being 30, this is a total of 242 (or 152)
observations!
While this sounds like quite a bit, in the world of machine learning it is not too
much.
A small 32 x 32 image grayscale image would have 1024 observations, which is already
considerably larger.
However, more observations means more processing time, and, when reinforcement
learning is used for robotics, additional sensors also mean increased cost.

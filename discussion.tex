This work explored a solution to the problem faced by bioengineers to determine those
cellular mechanisms to target for modification when improving vascularization for
synthetic tissue engineering. The method integrated a hybrid particle-based model of
vascularization as a fitness function with a genetic algorithm applied to optimize
the search space of model parameter values. The method is generic in that models of
different biological systems could be substituted for vascularization and the fitness
function likewise changed to reflect alternative objectives, such as minimizing
vascular function to combat tumor growth in cancer \cite{Mahoney2010MultiObjective}.

In this initial study, the search space was reduced to only consider parameters that
influence the chemotactic mechanism of the vascular cells. Previous work with the
model identified the importance of adhesion and tight junction formation in robust
network formation. However, increasing the number of parameters to include these
cellular mechanisms would significantly increase the search space, resulting in many
more fitness function calls. Given the modeling framework employed here
\cite{Lardon2011IDynoMiCS}, this is currently impractical because each fitness
evaluation takes at least four hours of CPU time running on a 3.6~GHz machine.
Recently two fast large-scale simulation systems have been developed by Ghaffarizadeh
et al. \cite{ghaffarizadeh2015agent} and \textsl{Biocellion}
\cite{delindavis:biocellion}. Both these systems implement an individual-based
approach similar to \textsl{cDynoMiCs} employed here. \textsl{Biocellion} is
implemented as a distributed architecture executable on the Cloud
\cite{Hashem2015Rise} and is capable of simulating complex 3D models of billions of
cells in a matter of a few hours. \textsl{Biocellion} has the potential to simulate
vascularization quickly enabling the optimization of more complex bioengineered
multicellular systems.

The principle challenge in bringing this work to practicality is to operationalize
the link between model parameters and actual genetic sequences. If this can be
achieved, then the output of the optimization process could be directly mapped to
engineering targets in the genome. In traditional bioengineering where products are
produced in biofactories \cite{delindavis:Sharma2001Production},
\cite{delindavis:vanDijl2013Bacillus}, targets are identified through analysis of
detailed metabolic models of microbial cells, such as \cite{Karp2015Pathway}. The
functioning of mammalian cells is less understood, particularly mechanisms such as
chemotaxis that are unrelated to metabolic processes, which can be modeled using flux
balance analysis.

The key to operationalizing bioengineering of multicellular systems is to utilize
multiscale models of the biological system under study. Multiscale models link the
mechanisms acting at different spatial and temporal scales together into an
integrated system where changes in expression and regulation of genes are manifested
in large-scale multicellular outcomes \cite{Martins2010Multiscale},
\cite{JosephWalpole2013Multiscale} \cite{Yu2016Multiclass}. If such detailed models
were employed, the high-level abstract parameters of this work would be replaced by
specific genetic mechanisms resulting in solutions being mapped directly to genetic
targets.


